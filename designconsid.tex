\section*{Design Considerations}
Two design issues that needed to be addressed in the filter design were data truncation and overflow/underflow handling.

\subsection*{Data Truncation}
The first design issue considered was data truncation. Data truncation is necessary because when two 8-bit numbers (such as the input and a filter coefficient) are multiplied together, the result is a 15-bit number. The filter design needs to handle this in some way. Possible options include discarding extra bits or rounding after every multiplier so that the circuit never has to handle more than 8 bits. Another approach is to carry the full 15-bit output of each multiplier to the final output of the filter and perform discarding/rounding only at the end.

In order to increase the accuracy of the filter, it was decided to utilize the second approach. The circuit is small enough that the added cost of carrying the extra bits to the final ouput is negligible. Essentially, the only additional circuitry required is a modification of two adders to be 16-bit adders instead of 8-bit adders and possibly 16-bit registers instead of 8-bit registers depending on the design utilized. 

Both discarding bits and rounding techniques were utilized in our design. We eventually settled on rounding because the obtained accuracy was desirable, especially given the negligible cost in both implementation and performance. The Design Details section discusses this in further detail.


\subsection*{Overflow/Underflow Handling}
The second design issue considered was overflow/underflow handling. All of the values handled by the filter are normalized to fall within a range of -1 to +1. However, because the data width of the filter is fixed, it may result that a value is generated by the filter that exceeds +1 or is below -1.

Our initial design addressed this with additional behavioral logic in the Verilog adder circuit. If overflow was detected, the adder would output +1. If underflow was detected, the adder would output -1. It was later determined that this logic had little effect on the accuracy of the circuit, and was subsequently discarded in favor of performance gains from a more-optimized adder circuit. The Design Details section discusses this in further detail.
